\chapter{Аналитический раздел}

\section{Постановка задачи}

В соответствии с техническим заданием на курсовую работу необходимо разработать загружаемый модуль ядра для получения информации о процессах, которые используют заданный пользователем файл. 
Также необходимо обеспечить возможность посылки сигналов рассматриваемым процессам. 
Для решения поставленной задачи необходимо:
\begin{enumerate}[label=\arabic*.]
	\item провести анализ структур и функций, предоставляющих возможность реализовать поставленную задачу;
	\item разработать алгоритмы и структуру загружаемого модуля ядра, в соответствии с поставленной задачей;
	\item реализовать спроектированный модуль ядра;
	\item протестировать работу реализованного модуля ядра.
\end{enumerate}

К разрабатываемой программе предъявляются следующие требования:
\begin{enumerate}[label=\arabic*.]
	\item взаимодействие с загружаемым модулем должно происходить из пространства пользователя;
	\item необходимо передавать данные из пространства ядра в пространство пользователя или наоборот.
\end{enumerate}

Программа будет разрабатываться для операционной системы Ubuntu~\cite{ubuntu} версии 20.04.4. 
В связи с этим она может не поддерживаться на более ранних версиях или на других операционных системах.

\section{Анализ структур ядра}

\subsection{struct task\_struct}

task\_struct --- структура ядра, описывающая процесс в операционной системе. 
Она содержит всю информацию, необходимую ядру для управления процессом.
В листинге \ref{listing_task} представлены необходимые для решения задачи фрагменты структуры task\_struct~\cite{task_struct}.

\begin{center}
	\begin{lstlisting}[label=listing_task,caption=Структура task\_struct]
struct task_struct {
// ...
unsigned int __state;
// ...
int prio;
// ...
pid_t pid;
// ...
struct task_struct __rcu *parent;
// ...
char comm[TASK_COMM_LEN];
// ...
struct files_struct *files;
// ...
	\end{lstlisting}
\end{center}

Подробное описание представленного фрагмента структуры task\_struct:
\begin{enumerate}
	\item \_\_state --- состояние процесса;
	\item prio --- приоритет процесса;
	\item pid --- идентификатор процесса;
	\item parent --- указатель на структуру родительского процесса;
	\item comm --- имя исполняемого файла;
	\item files --- указатель на структуру, содержащую информацию об открытых файлах.
\end{enumerate}

Для получения файловых дескрипторов процесса необходимо рассмотреть структуру files\_struct~\cite{lsc}.
В листингах \ref{listing_files_struct1}--\ref{listing_files_struct2} она полностью приведена.

\begin{center}
	\begin{lstlisting}[label=listing_files_struct1,caption=Структура files\_struct]
struct files_struct {
	atomic_t count;
	bool resize_in_progress;
	\end{lstlisting}
\end{center}

\begin{center}
	\begin{lstlisting}[label=listing_files_struct2,caption=Структура files\_struct]
	wait_queue_head_t resize_wait;
	
	struct fdtable __rcu *fdt;
	struct fdtable fdtab;

	spinlock_t file_lock ____cacheline_aligned_in_smp;
	unsigned int next_fd;
	unsigned long close_on_exec_init[1];
	unsigned long open_fds_init[1];
	unsigned long full_fds_bits_init[1];
	struct file __rcu * fd_array[NR_OPEN_DEFAULT];
};
	\end{lstlisting}
\end{center}

В данной структуре особого внимания заслуживает указатель fdt на структуру fdtable, именно в ней находится массив fd, содержащий структуры открытых файлов file для данного процесса.
Ниже представлен листинг структуры fdtable~\cite{lsc}.
\begin{center}
	\begin{lstlisting}[label=listing_fdtable,caption=Структура fdtable]
struct fdtable {
	unsigned int max_fds;
	struct file __rcu **fd;
	unsigned long *close_on_exec;
	unsigned long *open_fds;
	unsigned long *full_fds_bits;
	struct rcu_head rcu;
};
\end{lstlisting}
\end{center}

\subsection{struct file}

struct file~\cite{file} --- структура ядра, описывающая открытый файл. В листингах \ref{listing_file1}--\ref{listing_file2} полностью представлена данная структура.
\begin{center}
	\begin{lstlisting}[label=listing_file1,caption=Структура file]
struct file {
	union {
		struct llist_node	f_llist;
		struct rcu_head 	f_rcuhead;
\end{lstlisting}
\end{center}

\begin{center}
	\begin{lstlisting}[label=listing_file2,caption=Структура file]
		unsigned int 		f_iocb_flags;
	};
	
	spinlock_t		f_lock;
	fmode_t			f_mode;
	atomic_long_t		f_count;
	struct mutex		f_pos_lock;
	loff_t			f_pos;
	unsigned int		f_flags;
	struct fown_struct	f_owner;
	const struct cred	*f_cred;
	struct file_ra_state	f_ra;
	struct path		f_path;
	struct inode		*f_inode;
	const struct file_operations	*f_op;
	
	u64			f_version;
	#ifdef CONFIG_SECURITY
	void			*f_security;
	#endif
	void			*private_data;
	
	#ifdef CONFIG_EPOLL
	
	struct hlist_head	*f_ep;
	#endif
	struct address_space	*f_mapping;
	errseq_t		f_wb_err;
	errseq_t		f_sb_err;
};
	\end{lstlisting}
\end{center}

Для получения информации о том, какой файл представляет данная структура необходимо получить структуру dentry. Указатель на неё можно найти в структуре path, которая содержится в структуре file.
В листинге \ref{listing_path} представлена структура path~\cite{lsc}.

\pagebreak

\begin{center}
	\begin{lstlisting}[label=listing_path,caption=Структура path]
struct path {
	struct vfsmount *mnt;
	struct dentry *dentry;
};
\end{lstlisting}
\end{center}

\section{Анализ системных вызовов}

\subsection{Системный вызов kern\_path}

Пользователь загружаемого модуля будет передавать полный путь до файла, который он желает отследить, то есть найти все процессы, которые на момент запроса используют данный файл.
По этой причине необходимо получить по полному имени файла указатель на структуру dentry данного файла.
Данный функционал предоставляет системный вызов kern\_path, заголовок которого представлен в листинге \ref{listing_kern_path}.
\begin{center}
	\begin{lstlisting}[label=listing_kern_path,caption=Заголовок системного вызова kern\_path]
int kern_path(const char *name, unsigned int flags, struct path *path);
	\end{lstlisting}
\end{center}
На вход kern\_path принимает name --- имя файла; flags --- флаги поиска элемента пути; path --- структура path с результатом поиска.
Системный вызов возвращает 0 в случае успеха и код ошибки при неудаче.

\subsection{Системный вызов send\_sig\_info}

В соответствии с заданием необходимо предоставить возможность посылать сигналы процессам, использующим заданный файл.
Данный функционал предоставляет системный вызов send\_sig\_info, заголовок которого представлен в листинге \ref{listing_send_sig_info}.
\begin{center}
	\begin{lstlisting}[label=listing_send_sig_info,caption=Заголовок системного вызова send\_sig\_info]
int send_sig_info(int sig, struct siginfo *info, struct task_struct *p)
	\end{lstlisting}
\end{center}
На вход send\_sig\_info принимает sig --- номер сигнала; info --- структура, содержащая информацию о сигнале, возможно вместо структуры передать одно из двух числовых значений~\cite{send_signal}: 0, если сигнал послан из пространства пользователя, или 1, если сигнал послан из пространства ядра; p --- структура процесса, которому отправляется сигнал.
Системный вызов возвращает 0 в случае успеха и код ошибки при неудаче.

\subsection{Системный вызов copy\_from\_user}

В соответствии с требованиями необходимо передавать данные из пространства пользователя в пространство ядра.
Данный функционал предоставляет системный вызов copy\_from\_user, заголовок которого представлен в листинге \ref{listing_copy_from_user}.
\begin{center}
	\begin{lstlisting}[label=listing_copy_from_user,caption=Заголовок системного вызова copy\_from\_user]
long copy_from_user(void *to, const void __user *from, long n);
	\end{lstlisting}
\end{center}
На вход copy\_from\_user принимает to --- указатель на память в пространстве ядра, в которую будет осуществлено копирование; from --- указатель на память в пространстве пользователя, из которой будет осуществлено копирование; n --- размер копируемых данных.
Системный вызов возвращает 0 в случае успеха и ненулевое значение при неудаче.

\subsection{Системный вызов copy\_to\_user}

В соответствии с требованиями необходимо передавать данные из пространства ядра в пространство пользователя.
Данный функционал предоставляет системный вызов copy\_to\_user, заголовок которого представлен в листинге \ref{listing_copy_to_user}.
\begin{center}
	\begin{lstlisting}[label=listing_copy_to_user,caption=Заголовок системного вызова copy\_to\_user]
		long copy_to_user(void __user *to, const void *from, long n);
	\end{lstlisting}
\end{center}
На вход copy\_to\_user принимает to --- указатель на память в пространстве пользователя, в которую будет осуществлено копирование; from --- указатель на память в пространстве ядра, из которой будет осуществлено копирование; n --- размер копируемых данных.
Системный вызов возвращает 0 в случае успеха и ненулевое значение при неудаче.

\section{Интерфейс взаимодействия с модулем}


Для взаимодействия с модулем ядра из пространства пользователя будут использоваться файлы, созданные в /proc.
/proc --- интерфейс, предоставляющий доступ к структурам ядра. Передача информации будет достигатся при помощи привычных обращений чтения и записи к файлам.

\subsection{struct proc\_ops}

Для файлов, созданных в /proc сущетсвует специальная структура proc\_ops, содержащая указатели на функции взаимодействия с файлом, такие как открытие, закрытие, чтение и запись, заменяющая аналогичную структуру file\_operations для файлов на диске.
В листинге \ref{listing_proc_ops} представлена структура proc\_ops.

\begin{center}
	\begin{lstlisting}[label=listing_proc_ops,caption=Структура proc\_ops]
struct proc_ops {
	unsigned int proc_flags;
	int	(*proc_open)(struct inode *, struct file *);
	ssize_t	(*proc_read)(struct file *, char __user *, size_t, loff_t *);
	ssize_t (*proc_read_iter)(struct kiocb *, struct iov_iter *);
	ssize_t	(*proc_write)(struct file *, const char __user *, size_t, loff_t *);
	loff_t	(*proc_lseek)(struct file *, loff_t, int);
	int	(*proc_release)(struct inode *, struct file *);
	__poll_t (*proc_poll)(struct file *, struct poll_table_struct *);
	long	(*proc_ioctl)(struct file *, unsigned int, unsigned long);
	#ifdef CONFIG_COMPAT
	long	(*proc_compat_ioctl)(struct file *, unsigned int, unsigned long);
	#endif
	int	(*proc_mmap)(struct file *, struct vm_area_struct *);
	unsigned long (*proc_get_unmapped_area)(struct file *, unsigned long, unsigned long, unsigned long, unsigned long);
};
	\end{lstlisting}
\end{center}

Для изменения поведения файла при чтении или записи необходимо создать экземпляр структуры со своими функциями чтения и записи.



\subsection{системный вызов proc\_mkdir}

Системный вызов proc\_mkdir создаёт в /proc директорию.
Его заголовок представлен в листинге \ref{listing_proc_mkdir}.
\begin{center}
	\begin{lstlisting}[label=listing_proc_mkdir,caption=Заголовок системного вызова proc\_mkdir]
struct proc_dir_entry *proc_mkdir(const char *name, struct proc_dir_entry *parent);
	\end{lstlisting}
\end{center}
На вход proc\_mkdir принимает name --- имя создаваемой директории; parent --- указатель на структуру proc\_dir\_entry, описывающую родительскую директорию, если равно NULL, то директория создаётся в корне.
Системный вызов возвращает указатель на структуру proc\_dir\_entry созданной директории в случае успеха и NULL при неудаче.

\subsection{системный вызов proc\_mkdir}

Системный вызов proc\_create создаёт в /proc файл.
Его заголовок представлен в листинге \ref{listing_proc_create}.
\begin{center}
	\begin{lstlisting}[label=listing_proc_create,caption=Заголовок системного вызова proc\_create]
struct proc_dir_entry *proc_create(const char *name, umode_t mode, struct proc_dir_entry *parent,const struct file_operations *proc_fops);
	\end{lstlisting}
\end{center}
На вход proc\_create принимает name --- имя создаваемого файла; parent --- указатель на структуру proc\_dir\_entry, описывающую родительскую директорию, если равно NULL, то файл создаётся в корне; proc\_fops --- указатель на структуру с функциями работы с файлом.
Системный вызов возвращает указатель на структуру proc\_dir\_entry созданного файла в случае успеха и NULL при неудаче.

\subsection{системный вызов proc\_symlink}

Системный вызов proc\_symlink создаёт в /proc символическую ссылку.
Его заголовок представлен в листинге \ref{listing_proc_symlink}.
\begin{center}
	\begin{lstlisting}[label=listing_proc_symlink,caption=Заголовок системного вызова proc\_symlink]
struct proc_dir_entry *proc_symlink(const char *name,struct proc_dir_entry *parent,const char *dest);
	\end{lstlisting}
\end{center}
На вход proc\_symlink принимает name --- имя создаваемой символической ссылки; parent --- указатель на структуру proc\_dir\_entry, описывающую родительскую директорию, если равно NULL, то символическая ссылка создаётся в корне; dest --- имя файла, для которого создаётся символическая ссылка.
Системный вызов возвращает указатель на структуру proc\_dir\_entry созданной символической сслыки в случае успеха и NULL при неудаче.
