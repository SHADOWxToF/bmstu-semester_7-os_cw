\chapter{Технологический раздел}

\section{Выбор языка и среды программирования}

Для реализации ПО был выбран язык программирования Cи~\cite{c}, поскольку в нём есть все инструменты для реализации загружаемого модуля ядра. Средой программирования послужил графический редактор Visual Studio Code~\cite{vscode}, так как в нём много плагинов, улучшающих процесс разработки.

\section{Реализация загружаемого модуля}

В листингах \ref{listing_init1}--\ref{listing_init2} представлена функция загрузки модуля, а в листинге \ref{listing_exit} функция выгрузки модуля.

\begin{center}
	\begin{lstlisting}[label=listing_init1,caption=Функция загрузки модуля]
static int __init myinit(void)
{   
	command_info[0] = 0;
	if (alloc_proc_file_list(&proc_file_list, 10, 2))
	{
		return -ENOMEM;
	}
	if (!(main_dir = proc_mkdir(MAIN_DIR, NULL)))
	{
		free_proc_file_list(&proc_file_list);
		return -ENOMEM;
	}
	if (!(help_file = proc_create(HELP_FILE, 0444, main_dir, &help_ops)))
	{
		free_proc_file_list(&proc_file_list);
		remove_proc_entry(MAIN_DIR, NULL);
		return -ENOMEM;
	}
	if (!(command_file = proc_create(COMMAND_FILE, 0646, main_dir, &command_ops)))
	{
		free_proc_file_list(&proc_file_list);
	\end{lstlisting}
\end{center}

\begin{center}
	\begin{lstlisting}[label=listing_init2,caption=Функция загрузки модуля]
		remove_proc_entry(HELP_FILE, NULL);
		remove_proc_entry(MAIN_DIR, NULL);
		return -ENOMEM;
	}
	proc_file_list_append(&proc_file_list, MAIN_DIR, NULL);
	proc_file_list_append(&proc_file_list, HELP_FILE, main_dir);
	proc_file_list_append(&proc_file_list, COMMAND_FILE, main_dir);
	return 0;
}
	\end{lstlisting}
\end{center}

\begin{center}
\begin{lstlisting}[label=listing_exit,caption=Функция выгрузки модуля]
static void __exit myexit(void)
{
	proc_file_t *pfile;
	while ((pfile = proc_file_list_pop(&proc_file_list)))
	{ 
		remove_proc_entry(pfile->name, pfile->parent);
		kfree(pfile->name);
		kfree(pfile);
	}
	free_proc_file_list(&proc_file_list);
}
	\end{lstlisting}
\end{center}

В листингах \ref{listing_command_write1}--\ref{listing_command_write2} представлена функция записи в файл command.

\begin{center}
	\begin{lstlisting}[label=listing_command_write1,caption=Функция записи в файл command]
static ssize_t command_write(struct file *file, const char __user *ubuf, size_t count, loff_t *ppos) 
{
	if (dentry_library.len == MAX_LIBRARY_LEN)
	{
		sprintf(command_info, "Невозможно отслеживать более %d файлов\n", MAX_LIBRARY_LEN);
		return count;
	}
	char kbuf[10 * MAX_FILE_LEN + 1];
	if (copy_from_user(kbuf, ubuf, count))
	return -EFAULT;
	\end{lstlisting}
\end{center}

\begin{center}
	\begin{lstlisting}[label=listing_command_write2,caption=Функция записи в файл command]
	kbuf[count - 1] = 0;
	struct dentry *dentry = get_dentry_from_pathname(kbuf);
	struct proc_dir_entry *dir;
	struct proc_dir_entry *report_file;
	int status = 0;
	if (!dentry)
	sprintf(command_info, "Данного пути не существует в системе");
	else if (!(dir = proc_mkdir(dentry->d_name.name, main_dir)))
	status = -ENOMEM;
	else if (!(proc_symlink("symlink", dir, kbuf)))
	{
		remove_proc_entry(dentry->d_name.name, main_dir);
		status = -ENOMEM;
	}
	else if (!(report_file = proc_create("report", 0646, dir, &report_ops)))
	{
		remove_proc_entry("symlink", dir);
		remove_proc_entry(dentry->d_name.name, main_dir);
		status = -ENOMEM;
	}
	else if (status)
	sprintf(command_info, "Ошибка при создании директорий\n");
	else
	{
		proc_file_list_append(&proc_file_list, dentry->d_name.name, main_dir);
		proc_file_list_append(&proc_file_list, "symlink", dir);
		proc_file_list_append(&proc_file_list, "report", dir);
		dentry_library.array[(dentry_library.len)++] = dentry;
		sprintf(command_info, "Путь распознан. Создана папка %s, proc_dir_entry=%p\n", dentry->d_name.name, (void *) report_file);
	}
	return count;
}
	\end{lstlisting}
\end{center}

\pagebreak

В листингах \ref{listing_report_read1}--\ref{listing_report_read2} представлена функция чтения из файла report.

\begin{center}
	\begin{lstlisting}[label=listing_report_read1,caption=Функция чтения из файла report]
static ssize_t report_read(struct file *file, char __user *ubuf,size_t count, loff_t *ppos)
{
	if (*ppos > 0)
	return 0;
	
	int len = 0;
	char message[20 * MAX_FILE_LEN + 1];
	
	int i = 0;
	while (i < dentry_library.len && strcmp(dentry_library.array[i]->d_name.name, file->f_path.dentry->d_parent->d_name.name))
	++i;
	struct dentry *dentry = dentry_library.array[i];
	if (!dentry)
	{
		sprintf(command_info, "Что - то пошло не так\n");
		len += sprintf(message, "Что - то пошло не так\n");
	}
	else
	{
		sprintf(command_info, "Команда выполнена успешно\n");
		len += sprintf(message + len, "%7s %7s %7s %7s %7s %7s\n", "PPID", "PID", "FD", "PRIO", "STATE", "COMMAND");
		struct task_struct *task = &init_task;
		do
		{
			struct files_struct *files = task->files;
			struct fdtable *fdt = files->fdt;
			if (fdt && files)
			{
				for (int i = 0; i < files->next_fd; ++i)
				{
					struct file *f = (fdt->fd)[i];
					if (f->f_path.dentry == dentry)
					{
\end{lstlisting}
\end{center}

\begin{center}
	\begin{lstlisting}[label=listing_report_read2,caption=Функция чтения из файла report]
						len += sprintf(message + len, "%7d %7d %7d %7d %7d %s\n", task->parent->pid, task->pid, i, task->prio, task->__state, task->comm);
					}
				}
			}
		}
		while ((task = next_task(task)) != &init_task);
	}
	if (copy_to_user(ubuf, message, len))
	return -EFAULT;
	*ppos += len;
	return len;
}
	\end{lstlisting}
\end{center}

В листингах \ref{listing_report_write1}--\ref{listing_report_write2} представлена функция записи в файл report.

\begin{center}
	\begin{lstlisting}[label=listing_report_write1,caption=Функция записи в файл report]
static ssize_t report_write(struct file *file, const char __user *ubuf, size_t count, loff_t *ppos)
{
	int i = 0;
	while (i < dentry_library.len && strcmp(dentry_library.array[i]->d_name.name, file->f_path.dentry->d_parent->d_name.name))
	++i;
	struct dentry *dentry = dentry_library.array[i];
	if (!dentry)
	sprintf(command_info, "Что - то пошло не так\n");
	else
	{
		char kbuf[10 * MAX_FILE_LEN + 1];
		if (copy_from_user(kbuf, ubuf, count))
		{
			sprintf(command_info, "Не удалось скопировать данные из пользовательского режима\n");
			return -EFAULT;
		}
		kbuf[count - 1] = 0;
	\end{lstlisting}
\end{center}

\begin{center}
	\begin{lstlisting}[label=listing_report_write2,caption=Функция записи в файл report]		
		long sig;
		int res = kstrtol(kbuf, 10, &sig);
		if (res)
		{
			sprintf(command_info, "Введены некорректные данные\n");
			return res;
		}
		struct task_struct *task = &init_task;
		do
		{
			struct files_struct *files = task->files;
			struct fdtable *fdt = files->fdt;
			if (fdt && files)
			{
				int signalize = 0;
				for (int i = 0; !res && i < files->next_fd; ++i)
				{
					struct file *f = (fdt->fd)[i];
					if (f->f_path.dentry == dentry)
					{
						signalize = 1;
					}
				}
				if (signalize)
				res = send_sig_info(sig, 1, task);
			}
		}
		while (!res && (task = next_task(task)) != &init_task);
	}
	sprintf(command_info, "Команда выполнена успешно\n");
	return count;
}
	\end{lstlisting}
\end{center}

Для файлов были созданы экземпляры структуры proc\_ops, они представлены в листинге \ref{listing_proc_ops_c}.

\pagebreak

\begin{center}
	\begin{lstlisting}[label=listing_proc_ops_c,caption=Экземпляры структуры proc\_ops]		
static struct proc_ops help_ops = 
{
	.proc_read = help_read,
	.proc_open = myopen,
	.proc_release = myrelease,
};
static struct proc_ops command_ops = 
{
	.proc_read = command_read,
	.proc_write = command_write,
	.proc_open = myopen,
	.proc_release = myrelease,
};
static struct proc_ops report_ops = 
{
	.proc_read = report_read,
	.proc_write = report_write,
	.proc_open = myopen,
	.proc_release = myrelease,
};
\end{lstlisting}
\end{center}

Весь код программы представлен в Приложении А.

Для компиляции использовался makefile с наполнением, представленным в листингах \ref{listing_makefile1}--\ref{listing_makefile2}.
\begin{center}
	\begin{lstlisting}[label=listing_makefile1,caption=Makefile проекта]		
TARGET := finder
OBJS := main.o list.o
obj-m += $(TARGET).o
$(TARGET)-objs := $(OBJS)

KDIR ?= /lib/modules/$(shell uname -r)/build

ccflags-y += -std=gnu18 -Wall

all:
	make -C $(KDIR) M=$(shell pwd) modules
	\end{lstlisting}
\end{center}

\begin{center}
	\begin{lstlisting}[label=listing_makefile2,caption=Makefile проекта]		
$(TARGET).o: $(OBJS)
	$(LD) -r -o $@ $(OBJS)

reader:
	gcc -o reader.out reader.c

clean:
	make -C $(KDIR) M=$(shell pwd) clean
	\end{lstlisting}
\end{center}

