\chapter*{ПРИЛОЖЕНИЕ A}
\addcontentsline{toc}{chapter}{ПРИЛОЖЕНИЕ A}

\begin{center}
	\begin{lstlisting}[label=listing_main_c1,caption=Файл main.c]
#include <linux/module.h>
#include <linux/kernel.h>
#include <linux/proc_fs.h>
#include <linux/init.h>
#include <linux/path.h>
#include <linux/fs_struct.h>
#include <linux/fs.h>
#include <linux/namei.h>
#include <linux/file.h>
#include <linux/fdtable.h>
#include <linux/signal.h>
#include <linux/version.h>
#include <linux/uaccess.h>
#include <linux/slab.h>

#include "help.h"
#include "list.h"

#if LINUX_VERSION_CODE >= KERNEL_VERSION(5,6,0)
#define HAVE_PROC_OPS
#endif

#ifdef HAVE_PROC_OPS
static struct proc_ops myops;
static struct proc_ops help_ops;
static struct proc_ops command_ops;
static struct proc_ops report_ops;
#else
static struct file_operations myops;
static struct file_operations help_ops;
static struct file_operations command_ops;
static struct file_operations report_ops;
#endif

extern char *help_info;

MODULE_LICENSE("GPL");
	\end{lstlisting}
\end{center}

\begin{center}
	\begin{lstlisting}[label=listing_main_c2,caption=Файл main.c]
MODULE_AUTHOR("ALEX");
#define BUFSIZE PAGE_SIZE
#define HELPBUF 100

# define MAIN_DIR "finder"
static struct proc_dir_entry *main_dir;
# define HELP_FILE "help"
static struct proc_dir_entry *help_file;
# define COMMAND_FILE "command"
static struct proc_dir_entry *command_file;
static proc_file_list_t proc_file_list;

# define MAX_LIBRARY_LEN 20

static struct dentry_library
{
	int len;
	struct dentry *array[MAX_LIBRARY_LEN];
} dentry_library = {.len = 0};

static char command_info[10 * MAX_FILE_LEN + 1];

static struct dentry *get_dentry_from_pathname(const char *pathname) 
{
	struct path path;
	int error = kern_path(pathname, LOOKUP_FOLLOW, &path);
	if (!error)
	return path.dentry;
	else
	return NULL;
}

static ssize_t report_write(struct file *file, const char __user *ubuf, size_t count, loff_t *ppos)
{
	int i = 0;
	while (i < dentry_library.len && strcmp(dentry_library.array[i]->d_name.name, file->f_path.dentry->d_parent->d_name.name))
	++i;
	struct dentry *dentry = dentry_library.array[i];
	\end{lstlisting}
\end{center}

\begin{center}
\begin{lstlisting}[label=listing_main_c3,caption=Файл main.c]
	if (!dentry)
	sprintf(command_info, "Что-то пошло не так\n");
	else
	{
		char kbuf[10 * MAX_FILE_LEN + 1];
		if (copy_from_user(kbuf, ubuf, count))
		{
			sprintf(command_info, "Не удалось скопировать данные из пользовательского режима\n");
			return -EFAULT;
		}
		kbuf[count - 1] = 0;
		
		long sig;
		int res = kstrtol(kbuf, 10, &sig);
		if (res)
		{
			sprintf(command_info, "Введены некорректные данные\n");
			return res;
		}
		struct task_struct *task = &init_task;
		do
		{
			struct files_struct *files = task->files;
			struct fdtable *fdt = files->fdt;
			if (fdt && files)
			{
				int signalize = 0;
				for (int i = 0; !res && i < files->next_fd; ++i)
				{
					struct file *f = (fdt->fd)[i];
					if (f->f_path.dentry == dentry)
					{
						signalize = 1;
					}
				}
				if (signalize)
				res = send_sig_info(sig, 1, task);
			}
		}
	\end{lstlisting}
\end{center}

\begin{center}
\begin{lstlisting}[label=listing_main_c4,caption=Файл main.c]
		while (!res && (task = next_task(task)) != &init_task);
	}
	sprintf(command_info, "Команда выполнена успешно\n");
	return count;
}

static ssize_t report_read(struct file *file, char __user *ubuf,size_t count, loff_t *ppos)
{
	if (*ppos > 0)
	return 0;
	
	int len = 0;
	char message[20 * MAX_FILE_LEN + 1];
	
	int i = 0;
	while (i < dentry_library.len && strcmp(dentry_library.array[i]->d_name.name, file->f_path.dentry->d_parent->d_name.name))
	++i;
	struct dentry *dentry = dentry_library.array[i];
	if (!dentry)
	{
		sprintf(command_info, "Что-то пошло не так\n");
		len += sprintf(message, "Что-то пошло не так\n");
	}
	else
	{
		sprintf(command_info, "Команда выполнена успешно\n");
		len += sprintf(message + len, "%7s %7s %7s %7s %7s %7s\n", "PPID", "PID", "FD", "PRIO", "STATE", "COMMAND");
		struct task_struct *task = &init_task;
		do
		{
			struct files_struct *files = task->files;
			struct fdtable *fdt = files->fdt;
			if (fdt && files)
			{
				for (int i = 0; i < files->next_fd; ++i)
				{
					struct file *f = (fdt->fd)[i];
	\end{lstlisting}
\end{center}

\begin{center}
\begin{lstlisting}[label=listing_main_c5,caption=Файл main.c]
					if (f->f_path.dentry == dentry)
					{
						len += sprintf(message + len, "%7d %7d %7d %7d %7d %s\n", task->parent->pid, task->pid, i, task->prio, task->__state, task->comm);
					}
				}
			}
		}
		while ((task = next_task(task)) != &init_task);
	}
	if (copy_to_user(ubuf, message, len))
	return -EFAULT;
	*ppos += len;
	return len;
}

static ssize_t command_write(struct file *file, const char __user *ubuf, size_t count, loff_t *ppos) 
{
	if (dentry_library.len == MAX_LIBRARY_LEN)
	{
		sprintf(command_info, "Невозможно отслеживать более %d файлов\n", MAX_LIBRARY_LEN);
		return count;
	}
	char kbuf[10 * MAX_FILE_LEN + 1];
	if (copy_from_user(kbuf, ubuf, count))
	return -EFAULT;
	kbuf[count - 1] = 0;
	struct dentry *dentry = get_dentry_from_pathname(kbuf);
	struct proc_dir_entry *dir;
	struct proc_dir_entry *report_file;
	int status = 0;
	if (!dentry)
	sprintf(command_info, "Данного пути не существует в системе");
	else if (!(dir = proc_mkdir(dentry->d_name.name, main_dir)))
	status = -ENOMEM;
	else if (!(proc_symlink("symlink", dir, kbuf)))
	\end{lstlisting}
\end{center}

\begin{center}
\begin{lstlisting}[label=listing_main_c6,caption=Файл main.c]
	{
		remove_proc_entry(dentry->d_name.name, main_dir);
		status = -ENOMEM;
	}
	else if (!(report_file = proc_create("report", 0646, dir, &report_ops)))
	{
		remove_proc_entry("symlink", dir);
		remove_proc_entry(dentry->d_name.name, main_dir);
		status = -ENOMEM;
	}
	else if (status)
	sprintf(command_info, "Ошибка при создании директорий\n");
	else
	{
		proc_file_list_append(&proc_file_list, dentry->d_name.name, main_dir);
		proc_file_list_append(&proc_file_list, "symlink", dir);
		proc_file_list_append(&proc_file_list, "report", dir);
		dentry_library.array[(dentry_library.len)++] = dentry;
		sprintf(command_info, "Путь распознан. Создана папка %s, proc_dir_entry=%p\n", dentry->d_name.name, (void *) report_file);
	}
	return count;
}

static ssize_t command_read(struct file *file, char __user *ubuf,size_t count, loff_t *ppos) 
{
	if (*ppos > 0)
	return 0;
	
	int len;
	if (command_info[0])
	{
		len = strlen(command_info);
		if (copy_to_user(ubuf, command_info, len))
		return -EFAULT;
	}
	\end{lstlisting}
\end{center}

\begin{center}
\begin{lstlisting}[label=listing_main_c7,caption=Файл main.c]
	else
	{
		len = strlen(help_info);
		if (copy_to_user(ubuf, help_info, len))
		return -EFAULT;
	}
	*ppos += len;
	return len;
}

static ssize_t help_read(struct file *file, char __user *ubuf,size_t count, loff_t *ppos) 
{
	if (*ppos > 0)
	return 0;
	
	int len = strlen(help_info);
	if (copy_to_user(ubuf, help_info, len))
	return -EFAULT;
	*ppos += len;
	return len;
}

static int myopen(struct inode *inode, struct file *file)
{
	struct hlist_node *list = (inode->i_dentry).first;
	struct dentry *d = list_entry(list, struct dentry, d_u.d_alias);
	printk(KERN_INFO "file %s is opened", d->d_name.name);
	return 0;
}

static int myrelease(struct inode *inode, struct file *file)
{
	struct hlist_node *list = (inode->i_dentry).first;
	struct dentry *d = list_entry(list, struct dentry, d_u.d_alias);
	printk(KERN_INFO "file %s is released", d->d_name.name);
	return 0;
}
	\end{lstlisting}
\end{center}

\begin{center}
\begin{lstlisting}[label=listing_main_c8,caption=Файл main.c]
#ifdef HAVE_PROC_OPS
static struct proc_ops help_ops = 
{
	.proc_read = help_read,
	.proc_open = myopen,
	.proc_release = myrelease,
};
static struct proc_ops command_ops = 
{
	.proc_read = command_read,
	.proc_write = command_write,
	.proc_open = myopen,
	.proc_release = myrelease,
};
static struct proc_ops report_ops = 
{
	.proc_read = report_read,
	.proc_write = report_write,
	.proc_open = myopen,
	.proc_release = myrelease,
};
#else
static struct file_operations help_ops =
{
	.read = help_read,
	.open = myopen,
	.release = myrelease,
};
static struct file_operations command_ops =
{
	.read = command_read,
	.write = command_write,
	.open = myopen,
	.release = myrelease,
};
static struct file_operations report_ops =
{
	.read = report_read,
	.write = report_write,
	.open = myopen,
	\end{lstlisting}
\end{center}

\begin{center}
\begin{lstlisting}[label=listing_main_c9,caption=Файл main.c]
	.release = myrelease,
};
#endif

static int __init myinit(void)
{   
	command_info[0] = 0;
	if (alloc_proc_file_list(&proc_file_list, 10, 2))
	{
		return -ENOMEM;
	}
	if (!(main_dir = proc_mkdir(MAIN_DIR, NULL)))
	{
		free_proc_file_list(&proc_file_list);
		return -ENOMEM;
	}
	if (!(help_file = proc_create(HELP_FILE, 0444, main_dir, &help_ops)))
	{
		free_proc_file_list(&proc_file_list);
		remove_proc_entry(MAIN_DIR, NULL);
		return -ENOMEM;
	}
	if (!(command_file = proc_create(COMMAND_FILE, 0646, main_dir, &command_ops)))
	{
		free_proc_file_list(&proc_file_list);
		remove_proc_entry(HELP_FILE, NULL);
		remove_proc_entry(MAIN_DIR, NULL);
		return -ENOMEM;
	}
	proc_file_list_append(&proc_file_list, MAIN_DIR, NULL);
	proc_file_list_append(&proc_file_list, HELP_FILE, main_dir);
	proc_file_list_append(&proc_file_list, COMMAND_FILE, main_dir);
	return 0;
}

static void __exit myexit(void)
{
	proc_file_t *pfile;
	\end{lstlisting}
\end{center}

\begin{center}
\begin{lstlisting}[label=listing_main_c10,caption=Файл main.c]
	while ((pfile = proc_file_list_pop(&proc_file_list)))
	{ 
		remove_proc_entry(pfile->name, pfile->parent);
		kfree(pfile->name);
		kfree(pfile);
	}
	free_proc_file_list(&proc_file_list);
}

module_init(myinit);
module_exit(myexit);
	\end{lstlisting}
\end{center}

\begin{center}
	\begin{lstlisting}[label=listing_list_c1,caption=Файл list.c]
#include "list.h"

int alloc_proc_file_list(proc_file_list_t *list, int size, int k)
{
	list->array = kmalloc(size * sizeof(proc_file_t), GFP_KERNEL); // GFP_KERNEL - обычный запрос, который можно блокировать
	if (list->array)
	{
		list->k = k;
		list->size = size;
		list->len = 0;
		return 0;
	}
	return -1;
}

int proc_file_list_append(proc_file_list_t *list, const char *name, struct proc_dir_entry *pdentry)
{
	if (list->len == list->size)
	{
		proc_file_t *new_pointer = krealloc(list->array, list->size * list->k, GFP_KERNEL);
		if (!new_pointer)
		return -ENOMEM;
		list->array = new_pointer;
	\end{lstlisting}
\end{center}

\begin{center}
\begin{lstlisting}[label=listing_list_c2,caption=Файл list.c]
		list->size *= list->k;
	}
	char *new_name = kmalloc((MAX_FILE_LEN + 1) * sizeof(char), GFP_KERNEL);
	if (!new_name)
	return -ENOMEM;
	strcpy(new_name, name);
	list->array[list->len].name = new_name;
	list->array[list->len].parent = pdentry;
	++(list->len);
	return 0;
}

proc_file_t *proc_file_list_pop(proc_file_list_t *list)
{
	if (!list->len)
	return NULL;
	proc_file_t *result = kmalloc(sizeof(proc_file_t), GFP_KERNEL);
	if (!result)
	return NULL;
	--(list->len);
	memcpy(result, list->array + list->len, sizeof(proc_file_t));
	return result;
}

void free_proc_file_list(proc_file_list_t *list)
{
	kfree(list->array);
	list->array = NULL;
}
	\end{lstlisting}
\end{center}
