\chapter{Исследовательский раздел}

\section{Технические характеристики}

Технические характеристики устройства, на котором запускалась программа:

\begin{enumerate}[label=\arabic*.]
	\item операционная система Ubuntu, 20.04.4 \cite{ubuntu};
	\item память 8 ГБ;
	\item процессор 2,4 ГГц 4‑ядерный процессор Intel Core i5-1135G7 \cite{intel}.
\end{enumerate}

\section{Исследование работы программы}

Для исследования работы была разработана вспомогательная программа, которая открывает 10 раз один и тот же файл, а потом прерывается в ожидании ввода. Код вспомогательной программы представлен в листинге \ref{listing_help_program1}--\ref{listing_help_program2}.

\begin{center}
	\begin{lstlisting}[label=listing_help_program1,caption=Вспомогательная программа]
# include <stdio.h>
# include <signal.h>

int flag = 0;

void handler(int s)
{
	flag = 1;
}

int main()
{
	signal(SIGINT, handler);
	FILE *fd[10];
	for (int i = 0; i < 10; ++i)
		fd[i] = fopen("qwerty.txt", "r");
	char c;
	scanf("%c", &c);
	for (int i = 0; i < 10; ++i)
	\end{lstlisting}
\end{center}

\begin{center}
	\begin{lstlisting}[label=listing_help_program2,caption=Вспомогательная программа]
		fclose(fd[i]);
	if (flag)
		printf("Процесс принял сигнал SIGINT\n");
	return 0;
}
	\end{lstlisting}
\end{center}

На рисунке \ref{} продемонстрирована работа загружамого модуля:
\begin{enumerate}[label=\arabic*.]
	\item sudo insmod finder.ko --- загрузка модуля;
	\item cd /proc/finder --- переход в рабочую директорию модуля;
	\item echo "/home/alex/Bomonka/Semester\_7/Operating\_systems/bmstu-semester\_7-os\_cw/src/qwerty.txt" $>$ command --- создание папки для отслеживания заданного файла;
	\item cd qwerty.txt --- переход в папку отслеживания;
	\item cat report --- чтение файла report;
	\item echo "2" $>$ report --- посылка сигнала процессам;
	\item sudo rmmod finder --- выгрузка модуля.
\end{enumerate}

На рисунке \ref{} продемонстрирован приём сигнала вспомогательной программой.